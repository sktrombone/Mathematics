\documentclass[11pt,a4paper]{jsarticle}
\usepackage{amsthm}
\usepackage{amsmath}
\usepackage{latexsym}
\begin{document}

\newtheorem{theo}{Thm.}[section]
\newtheorem{defi}{Def.}[section]
\newtheorem{lemm}{Lem.}[section]
\newtheorem{prop}{Prop.}[section]
\newtheorem{ex}{Ex.}[section]
\newtheorem{prf}{Prf.}

\section{準同型定理と同型定理}

まず,復習として準同型定理を以下に示す.
\begin{theo}{\bf 準同型定理} \\
$群Gから群G'への準同型写像f:G\rightarrow G'に対して,N={\rm Ker}(f)$とすれば,$$\overline{f}:G/{\rm Ker}(f)\rightarrow {\rm Im}(f),xN \mapsto f(x)$$は同型写像となる.すなわち,剰余群G/{\rm Ker}(f)とfによるGの像{\rm Im}(f)は同型となる.$$\overline{f}:G/{\rm Ker}(f) \xrightarrow{\sim} {\rm Im}(f).$$
\end{theo}
次に標準全射を定義しておく.
\begin{defi}{\bf 標準全射} \\
$N\lhd G$に対して,全準同型写像$$f:G \rightarrow G/N,g\mapsto gN$$
を{\bf 標準全射}という.
\end{defi}

次に3つの同型定理を示す.
\begin{theo}{\bf 第1同型定理} \\
$群Gから群G'への全射準同型写像f:G\rightarrow G'と正規部分群N'\lhd G'に対して,N:=f^{-1}(N')\lhd G$かつ,$$G/N\simeq G'/N'.$$
\end{theo}

\begin{proof}
$fと標準全射\varphi との合成\varphi \circ f:G\rightarrow G' \rightarrow G'/N'$は全射同型であり,${\mathrm Ker}(\varphi \circ f) = \{x \in G | (\varphi \circ f)(x) = N' \}であるから,x\in {\mathrm Ker}(\varphi \circ f) \Longleftrightarrow \varphi (f(x))=N' \Longleftrightarrow f(x) \in N' \Longleftrightarrow x \in N.よって,{\mathrm Ker}(\varphi \circ f)=N \lhd Gで準同型定理より\overline{\varphi \circ f}:G/N \xrightarrow{\sim} G'/N'.$
\end{proof}
\begin{lemm}
$S,T,U\subset Gに対して,(ST)U=S(TU),(ST)^{-1}=T^{-1}S^{-1}.$
\end{lemm}

\begin{proof}
$Gが群であるから,(ST)U=\{(st)u\  |\  s\in S,t \in T,u \in U \}=\{s(tu)\  |\  s\in S,t \in T,u \in U \}=S(TU),(ST)^{-1}=\{(st)^{-1}\  |\  s\in S,t \in T,u \in U \}=\{t^{-1}s^{-1}\  |\  s\in S,t \in T,u \in U \}=T^{-1}S^{-1}.$
\end{proof}

\begin{lemm}
(1)$H,K \leq Gに対して,HK\leq G \Longleftrightarrow HK=KH.$ \\
(2)$H\leq GとN\lhd Gに対して,HN=NH\leq G.$
\end{lemm}

\begin{proof}
(1)$H,K\leq GよりH=H^{-1},K=K^{-1},HH=H,KK=K.(\Rightarrow)HK\leq GならばHK=(HK)^{-1}=K^{-1}H^{-1}=KH.(\Leftarrow)HK=KHならば,(HK)(HK)=H(KH)K=H(HK)K=(HH)(KK)=HK.よって,任意のhk.h'k'\in HKに対して,(hk)(h'k'),(hk)^{-1} \in HK.部分群の判定条件より,HK\leq G.$ \\
(2)$N\lhd Gより,h^{-1} Nh =N(\forall h \in H)であるから,hN = Nh (\forall h \in H)よりHN=NH.よって(1)よりHN\leq G.$
\end{proof}

\begin{theo}{\bf 第2同型定理} \\
$群Gの部分群H \leq GとN \lhd Gに対して,$ $$H/(H\cap N) \simeq HN/N.$$
\end{theo}

\begin{proof}
${\rm Lem.1.2.}(2)より,HN\leq G.また,N\lhd GよりN\leq HN.全射準同型f:H \rightarrow HN/N ,h \mapsto hN に対して,{\mathrm Ker}(f)=\{x \in H\  |\  xN=N \}であり,x \in {\mathrm Ker}(f) \Longleftrightarrow x \in H \cap Nより,{\mathrm Ker}(f)=H\cap N \lhd H.よって,準同型定理より,\overline{f}:H/(H\cap N) \xrightarrow{\sim} HN/N.$
\end{proof}

\begin{theo}{\bf 第3同型定理} \\
$群G とN_1,N_2 \lhd G,N_2\leq N_1に対して,$ $$(G/N_2)/(N_1/N_2)\simeq G/N_1.$$
\end{theo}

\begin{proof}
$f:G/N_2 \rightarrow G/N_1,xN_2 \mapsto xN_1は{\rm well-defined}かつ全射準同型となる.{\rm well-defined}であることは,N_2 \leq N_1 から,xN_2=yN_2 \Longleftrightarrow y^{-1}x \in N_2 \leq N_1 \Rightarrow y^{-1}x \in N_1 \Rightarrow (y^{-1})N_1 = N_1 \Rightarrow xN_1 =yN_1のようにわかる.準同型であることは,f(xN_2)f(yN_2)より分かり,全射もよい.また,{\mathrm Ker}(f)=\{xN_2 \in G/N_2 \  |\  f(xN_2)=N_1 \} = \{xN_2 \in G/N_2 \  |\  xN_1 =N_1 \} = \{xN_2 \in G/N_2\  |\  x \in N_1 \} = N_1/N_2 \lhd G/N_2 であるから,準同型定理より\overline{f}:(G/N_2)/(N_1/N_2) \xrightarrow{\sim} G/N_1.$
\end{proof}

\section{外部直積と内部直積}

\begin{defi}{\bf 外部直積,外部直和}
$G_1,\cdots ,G_n を群とする.直積集合$ $$G=G_1 \times \cdots \times G_n =\{ (x_1,\cdots .x_n)\  |\  x_i \in G_i\}$$
は積$(x_1,\cdots ,x_n)\circ (y_1,\cdots ,y_n):=(x_1 y_1,\cdots ,x_n y_n)で群をなし,G_1,\cdots ,G_nの{\bf 外部直積}といい,同じくG_1\times \cdots \times G_nとかく.各G_iをGの{\bf 直積因子}という.特に,各G_iが加群(G_i ,\sum)のときには,G_1,\cdots G_nの{\bf 外部直和}といい,G_1 \oplus \cdots \oplus G_nとかいて,各G_iをGの{\bf 直和因子}という.$
\end{defi}

\end{document}
